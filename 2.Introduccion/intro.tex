Este informe presenta el desarrollo del taller dedicado a la introducción de los conceptos de morfología en robots, específicamente en un brazo robótico. La actividad se centró en aspectos de los grados de libertad, tipos de articulaciones y características esenciales de los robots manipuladores. Para ello, se utilizó como referencia el brazo SCORBOT-ER 4u, así como recursos de apoyo como videos en la web.  

El taller se divide en tres partes:

\begin{enumerate}
    \item \textbf{Parte I:} Análisis del Brazo Robótico SCORBOT-ER 4u.
    
    Se trata de investigaron las características del brazo SCORBOT-ER 4u, incluyendo su diseño, funcionalidad y aplicaciones

    \item \textbf{Parte II:} Discusión de conceptos fundamentales sobre la morfología de robots.

    Esta parte consiste de una serie de preguntas relacionadas con los conceptos de grados de libertad, resolución, precisión, repetibilidad, volumen de trabajo y capacidad de carga de un robot industrial. 

    \item \textbf{Parte III:} Aplicación de los conceptos a casos prácticos.
    
    Se ven términos como reductores, accionamiento directo, actuadores,sensores y elementos terminales.
  
\end{enumerate}